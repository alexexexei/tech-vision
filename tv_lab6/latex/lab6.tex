\documentclass[a4paper, 12pt]{article}
\usepackage[utf8x]{inputenc}
\usepackage{cmap}
\usepackage[english, russian]{babel}
\usepackage{indentfirst}
\usepackage[left=20mm, top=20mm, right=20mm, bottom=20mm]{geometry}
\usepackage{tikz}
\usepackage{float}
\usepackage{amsmath, amsfonts, amssymb}
\usepackage{graphicx}
\usepackage{fancybox, fancyhdr}
\usepackage{hyperref}
\usepackage{listings}
\usepackage{caption}
\usepackage{subcaption}
\usepackage{xcolor}
\pagestyle{fancy}
\fancyhf{}
\fancyhead[L]{Лабораторная работа №6}
\fancyhead[R]{Техническое зрение}
\fancyfoot[C]{\thepage}
\graphicspath{{images/}}
\usetikzlibrary{patterns}
\definecolor{LightGray}{gray}{0.95}
\definecolor{LightGray2}{gray}{0.7}
\lstdefinestyle{pycode}{
    language=Python,
    basicstyle=\footnotesize\ttfamily,
    numbers=left,
    numberstyle=\scriptsize\color{gray},
    stepnumber=1,
    numbersep=5pt,
    backgroundcolor=\color{LightGray},
    showspaces=false,
    showstringspaces=false,
    showtabs=false,
    tabsize=4,
    captionpos=b,
    breaklines=true,
    breakatwhitespace=false,
    frame=single,
    rulecolor=\color{LightGray2},
    linewidth=\linewidth,
    keywordstyle=\color{blue}\bfseries,
    commentstyle=\color{green!40!black},
    stringstyle=\color{purple},
    escapeinside={\%*}{*)},
    inputencoding=utf8x,
    xleftmargin=0pt,
    framexleftmargin=0pt,
    framexrightmargin=0pt
}
\lstset{style=pycode}
\hypersetup{
    colorlinks=true,
    linkcolor=blue,
    filecolor=magenta,
    urlcolor=cyan,
    pdftitle={contents setup},
    pdfpagemode=FullScreen,
}
\setlength{\parskip}{1.5mm}
\setlength{\headheight}{15pt}
\setlength{\footskip}{15pt}
\allowdisplaybreaks
\DeclareMathOperator{\sinc}{sinc}
\newcommand{\frc}[2]{\raisebox{2pt}{$#1$}\big/\raisebox{-3pt}{$#2$}}

\begin{document}
    \begin{titlepage}

        \begin{center}
        \includegraphics[width=0.3\textwidth]{itmo.png} % requires itmo.png in /images folder
        \vfill

        Федеральное государственное автономное образовательное учреждение высшего образования
        «Национальный Исследовательский Университет ИТМО»\\

        \vfill
        {\large\bf ЛАБОРАТОРНАЯ РАБОТА №6}\\
        {\large\bf ПРЕДМЕТ «ТЕХНИЧЕСКОЕ ЗРЕНИЕ»}\\
        {\large\bf ТЕМА «МОРФОЛОГИЧЕСКИЙ АНАЛИЗ ИЗОБРАЖЕНИЙ»}
        \vfill

        \begin{flushright}
            \begin{minipage}{.45\textwidth}
            {
                \hbox{Преподаватель:}
                \hbox{Шаветов C. В.}
                \hbox{}
                \hbox{Выполнили:} 
                \hbox{Румянцев А. А.}
                \hbox{Чебаненко Д. А.}
                \hbox{Овчинников П. А.}
                \hbox{}
                \hbox{Поток: ТЕХ. ЗРЕНИЕ 2.1}
                \hbox{Факультет: СУиР}
                \hbox{Группа: R3241}
            }
            \end{minipage}
        \end{flushright}

        \vfill

        Санкт-Петербург\\
        2024
        \end{center}
    \end{titlepage}

    \tableofcontents

    \newpage
    \section{Цель работы}
    Освоение принципов математической морфологии в области обработки и анализа изображений.

    
    \section{Теоретические сведения}
    Математическая морфология в обработке изображений применяется для фильтрации шумов, сегментации
    объектов, выделения контуров, поиска заданного объекта на изображении, вычисления "скелета" образа и
    других преобразований. Далее рассмотрим базовые морфологические операции над изображением $A$ структурным
    элементом $B$.


    \section{Задание 1}
    Выберем произвольное изображение, содержащее дефекты формы (внутренние <<дырки>> или внешние <<выступы>> объектов).
    Пусть это будет мухомороподобный перченый флаг Японии, нарисованный в Paint. Так как фон изображения белого цвета,
    добавим рамку вокруг картинки для видимости границ.
    \begin{figure}[H]
        \centering
        \fbox{\includegraphics[scale=0.365]{base_morf.png}}
        \captionsetup{skip=0pt}
        \caption{Изображение для задания 1}
        \label{fig:izt1}
    \end{figure}
    Далее будем пробовать различные морфологические операции для устранения черных точек, белых точек и красных выступов.


    \subsection{Дилатация}
    Дилатация (расширение, наращивание): $A\oplus B$, расширяет бинарный образ $A$ структурным элементом $B$. Данная операция
    увеличивает белые области на изображении. Применим ее к исходной картинке и посмотрим результат.
    \begin{figure}[H]
        \centering
        \fbox{\includegraphics[scale=0.27]{dil_bm.png}}
        \captionsetup{skip=0pt}
        \caption{Применение дилатации к исходному изображению}
        \label{fig:dil1}
    \end{figure}
    Видим, что белые <<дырки>> внутри красного круга стали больше, черные меньше. Радиус окружности уменьшился, частично
    срезались выступы по краям.


    \subsection{Эрозия}
    Эрозия (сжатие, сужение): $A\ominus B$, сужает бинарный образ $A$ структурным элементом $B$. Эта операция уменьшает
    белые области на изображении. Применим ее к оригинальной картинке и сравним с ней результат.
    \begin{figure}[H]
        \centering
        \fbox{\includegraphics[scale=0.27]{er_bm.png}}
        \captionsetup{skip=0pt}
        \caption{Применение эрозии к исходному изображению}
        \label{fig:er1}
    \end{figure}
    Можем заметить, что белые <<дырки>> внутри красного круга пропали, черные увеличились. Радиус круга и дефекты по его краям
    стали больше. Наблюдаем эффект, обратный дилатации. Результат зависит от количества итераций, то есть от того, сколько раз
    операция была применена к изображению. Если выбрать меньшее количество итераций в данном пункте, то белые <<дырки>> уменьшатся,
    но не пропадут. То же самое в пункте про дилатацию -- если сделать больше итераций, то черные <<дырки>> полностью исчезнут.


    \subsection{Открытие}
    Открытие (отмыкание, размыкание, раскрытие): $(A\ominus B)\oplus B$, удаляет внешние дефекты бинарного образа $A$ структруным
    элементом $B$. Состоит из последовательного применения эрозии и дилатации. Применим операцию к изначальному изображению.
    \begin{figure}[H]
        \centering
        \fbox{\includegraphics[scale=0.27]{op_bm.png}}
        \captionsetup{skip=0pt}
        \caption{Применение открытия к исходному изображению}
        \label{fig:op1}
    \end{figure}
    Видим, что белые <<дырки>> пропали, черные остались нетронуты. Радиус красной окружности не изменился. В местах
    выступов круг начал сливаться с ними, что означает, что операция сгладила внешние дефекты.


    \subsection{Закрытие}
    Закрытие (замыкание): $(A\oplus B)\ominus B$, удаляет внутренние дефекты бинарного образа $A$ структурным элементом $B$.
    Состоит из последовательного применения дилатации и эрозии. Применим данную операцию к исходному изображению и посмотрим
    результат.
    \begin{figure}[H]
        \centering
        \fbox{\includegraphics[scale=0.27]{cl_bm.png}}
        \captionsetup{skip=0pt}
        \caption{Применение закрытия к исходному изображению}
        \label{fig:cl1}
    \end{figure}
    Черные <<дырки>> внутри круга исчзели, белые почти не изменились. Внешние дефекты окружности немного срезались, но
    вместе с некоторыми ее частями.


    \subsection{Комбинации}
    Попробуем сначала применить 4 раза эрозию, потом 9 раз дилатацию. Также применим последовательно открытие и закрытие. Посмотрим на результат.
    \begin{figure}[H]
        \centering
        \fbox{\includegraphics[scale=0.27]{er_then_dil_bm.png}}
        \captionsetup{skip=0pt}
        \caption{Применение эрозии и дилатации к исходному изображению}
        \label{fig:erdil1}
    \end{figure}
    \begin{figure}[H]
        \centering
        \fbox{\includegraphics[scale=0.27]{op_then_cl_bm.png}}
        \captionsetup{skip=0pt}
        \caption{Применение открытия и закрытия к исходному изображению}
        \label{fig:opcl1}
    \end{figure}
    Как видим, оба варианта выглядят неплохо. Пропали внутренние дефекты, внешние только сглажены, так как убрать их
    полностью без потери большей части информации о рассматриваемом объекте (красном круге) не получается (радиус окружности
    становится либо сильно больше, либо сильно меньше оригинала).


    \section{Задание 2}
\end{document}